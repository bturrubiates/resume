\documentclass[10pt,letterpaper,sans]{moderncv}

\moderncvstyle{banking}
\moderncvcolor{blue}
\usepackage[utf8]{inputenc}
\usepackage[scale=0.77]{geometry}
\usepackage{microtype}

\usepackage{import}

\name{Benjamin}{Turrubiates}
\phone[mobile]{956 802 9037}
\email{ben@turrubiat.es}
\social[github][github.com/bturrubiates]{bturrubiates}

\renewcommand*{\cventry}[7][.25em]{
  \begin{tabular*}{\textwidth}{l@{\extracolsep{\fill}}r}%
	  {\bfseries #4} & {\bfseries #5} \\%
	  {\itshape #3\ifthenelse{\equal{#6}{}}{}{#6}} & {\itshape #2}\\%
  \end{tabular*}%
  \ifx&#7&%
    \else{\\\vbox{\small#7}}\fi%
  \par\addvspace{#1}}

\begin{document}
\makecvtitle
\vspace{-2em}

\section{Employment}
\vspace{3pt}

\begin{itemize}
    \item{
        \cventry{
            January 2015--Present
        }{
            Post Baccalaureate Research Assistant, HPC-5
        }{
            Los Alamos National Laboratory
        }{
            Los Alamos, NM
        }{}{
            \vspace{3pt}
            \begin{itemize}
                \item {
                    Ported libfabric, a set of APIs for direct network hardware
                    access, to OS X.
                }
                \item {
                    Contributed a logging subsystem - allowing for fine grained
                    control over log levels, subsystems, and providers -- to
                    libfabric.
                }
                \item {
                    Currently contributing to the development of a libfabric
                    provider for the Cray Aires Generic Network Interface
                    (GNI).
                }
            \end{itemize}
        }
    }

    \item{
        \cventry{
            July 2014--December 2014
        }{
            Undergraduate Research Assistant, HPC-3
        }{
            Los Alamos National Laboratory
        }{
            Los Alamos, NM
        }{}{
            \vspace{3pt}
            \begin{itemize}
                \item {
                    Used C++ and the Component Based Tool Framework, a scalable
                    tree-based transport system, to develop parallel tools to
                    aggregate configuration information about a computing
                    cluster.
                }
                \item {
                    Contributed to the development of Pavilion, a customizable
                    software framework for launching and analyzing tests
                    targeting HPC platforms.
                }
            \end{itemize}
        }
    }

    \vspace{3pt}

    \item{
        \cventry{
            Summer 2013
        }{
            Undergraduate Research Assistant, HPC-3
        }{
            Los Alamos National Laboratory
        }{
            Los Alamos, NM
        }{}{
            \vspace{3pt}
            \begin{itemize}
                \item {
                    Developed a simple test harness which allows for easily
                    varying execution environments and test parameters in an
                    HPC environment using Python and Moab.
                }
            \end{itemize}
        }
    }

    \vspace{3pt}

    \item{
        \cventry{
            2011 -- 2012
        }{
            Outreach Educator
        }{
            New Mexico EPSCoR
        }{
            Socorro, NM
        }{}{
            \vspace{3pt}
            \begin{itemize}
                \item {
                    Advised middle and high school students on projects for the
                    New Mexico Supercomputing Challenge by providing feedback
                    on project proposals.
                }
                \item {
                    Worked with a team to develop curriculum for high school AP
                    Computer Science classes focusing on the Java programming
                    language.
                }
            \end{itemize}
        }
    }

\end{itemize}

\section{Presentations}

\vspace{2pt}

\begin{itemize}
    \item{
        \cventry{
            August 2014
        }{
            Presented at LANL Student Symposium
        }{
            Auto-Generated ClusterWide Configuration Tool
        }{
            LA-UR-14-25622
        }{
        }{
        }
    }
    \item{
        \cventry{
            August 2013
        }{
            Presented at HPC Mini-Showcase
        }{
            Compiler Performance Benchmark and Testing
        }{
            LA-UR-13-25887
        }{
        }{
        }
    }
\end{itemize}

\section{Projects}

\vspace{2pt}

\begin{itemize}
    \item {
        \cventry {
            Fall 2014
        }{
            C89 optimizing compiler
        }{
            \textit{MORT}
        }{
            Compiler Writing Semester Project
        }{}{
            \vspace{3pt}
            Developed an optimizing C89 compiler written in C++11 which uses a
            graph-based intermediate representation in SSA form -- a property
            that requires each variable is only assigned once -- to target
            x86-64 while performing common optimizations such as useless code
            elimination, and loop optimizations. Worked in a four person group
            using pair programming, continuous integration, a branch-heavy git
            workflow.
        }
    }

    \vspace{3pt}

    \item {
        \cventry {
            Spring 2014
        }{
            An actor model of type analysis
        }{
            \textit{Paratype}
        }{
            Parallel Programming Group Project
        }{}{
            \vspace{3pt}
            Developed a proof of concept parallel actor model of type analysis
            implemented in Google Go. \textit{Paratype} operates on a toy
            functional language that supports parametric polymorphism inspired
            by Haskell. We developed an algorithm which utilizes goroutines --
            concurrent function primitives in Go -- to reduce polymorphic code
            to monomorphic code by producing type-specific implementations.
        }
    }
\end{itemize}

\section{Education}

\vspace{2pt}

\begin{itemize}

    \item{
        \cventry{
            August 2010 -- December 2014
        }{
            B.Sc. Computer Science; minor in Electrical Engineering
        }{
            New Mexico Institute of Mining and Technology
        }{
            Socorro, NM
        }{
            \textit{GPA: 3.61}
        }{
        }
    }

\end{itemize}

\section{Relevant Coursework}

\vspace{2pt}

\begin{itemize}
    \item{
        \textbf{Completed:}
        \textit{
            Compiler Writing, Design and Analysis of Algorithms, Operating
            Systems, Parallel Programming, Software Engineering, Formal
            Languages and Automata, Computer Architecture, Systems Programming
        }
    }
\end{itemize}

\section{Skills}

\vspace{2pt}

\begin{itemize}

    \item{
        \textbf{Programming Languages:}
        \textit{
            C, Python, Clojure, Go, Java,  x86 assembly, C++, Shell (Bash, Zsh)
        }
    }
    \vspace{3pt}
    \item{
        \textbf{HPC Languages, Introductory:}
        \textit{
            OpenMP, MPI
        }
    }
    \vspace{3pt}
    \item{
        \textbf{Software:}
        \textit{
            Git, Vim, \LaTeX, CMake, GNU Make, Autotools
        }
    }
    \vspace{3pt}
    \item {
        \textbf{Data Formats:}
        \textit{
            JSON, YAML, XML
        }
    }
\end{itemize}

\section{Professional Memberships}
\begin{itemize}
    \item {
            Member; Association for Computing Machinery
    }
\end{itemize}

\end{document}


%% end of file `template.tex'.

